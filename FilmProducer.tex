\chapter{Film Producer}
\begin{multicols}{2}

Producers are in control of every aspect of a film’s production, including putting together and approving the whole production team. They act as the leaders who create a productive and successful work environment for everyone involved in making a film. Producers must produce, manufacture, and distribute movies, and they’re tasked with making sure their films are successful both during and after showings. An average day on the job involves moving around the working environment and making sure everything's running smoothly. Producers travel quite often during the work day, and must endure long car and plane rides in order to create a film. Most working hours are long and irregular and are usually determined by how much work the film needs. There are many skills that are good to have as a producer. Producers need to be able to work well with others, motivating and communicating with them to get the project done. Having good communication and organizing skills also complement the job of a producer. According to Truity, “The median annual wage for producers and directors was \$71,350 in May 2012. The lowest 10 percent earned less than \$32,080, and the top 10 percent earned more than \$187,200 in May 2012” (Truity). Most producers are self employed, so their salaries vary.

\end{multicols}

\section{Top Colleges}

\begin{table}[H]
\centering
\caption{Undergraduate Colleges}
\label{Film Producer Undergraduate Colleges}
\resizebox{\textwidth}{!}{%
\begin{tabular}{llrlr}
\hline
\multicolumn{5}{|l|}{Dream Schools}                                                             \\ \hline
University of Southern California & Total 4 year cost: & \$268,848 & 20 year ROI: & \$1,158,152 \\
New York University               & Total 4 year cost: & \$281,776 & 20 year ROI: & \$1,145,224 \\ \hline
\multicolumn{5}{|l|}{Best Value Schools}                                                        \\ \hline
UT Austin                         & Total 4 year cost: & \$105,384 & 20 year ROI: & \$1,321,616 \\
American University               & Total 4 year cost: & \$239,552 & 20 year ROI: & \$1,187,448
\end{tabular}%
}
\end{table}


\begin{table}[H]
\centering
\caption{Graduate Universities}
\label{Film Producer Graduate Universities}
\resizebox{\textwidth}{!}{%
\begin{tabular}{llrlr}
\hline
\multicolumn{5}{|l|}{Dream Schools}                                                                     \\ \hline
American Film Institute               & Total {[}{]} year cost: & \$76,273 & 20 year ROI: & \$1,350,727 \\
Columbia University                   & Total {[}{]} year cost: & \$65,860 & 20 year ROI: & \$1,361,140 \\ \hline
\multicolumn{5}{|l|}{Best Value Schools}                                                                \\ \hline
University of California, Los Angeles & Total {[}{]} year cost: & \$60,744 & 20 year ROI: & \$1,366,256 \\
San Francisco State University        & Total {[}{]} year cost: & \$38,808 & 20 year ROI: & \$1,391,192
\end{tabular}%
}
\end{table}

\begin{multicols}{2}
% \begin{itemize}
%     \item{Undergraduate} 
%     \begin{itemize}
%         \item{Dream Schools}
%             \begin{itemize}
%                 \item{University of Southern California}
%                     \begin{itemize}
%                         \item{Total 4 year cost (state):} \$268,848 
%                         \item{20 year ROI:} \$1,158,152 
%                     \end{itemize}
%                 \item{New York University}
%                     \begin{itemize}
%                         \item{Total 4 year cost (state):} \$281,776 
%                         \item{20 year ROI:} \$1,145,224
%                     \end{itemize}
%                 \item{American Film Institute}
%                     \begin{itemize}
%                         \item{Total [] year cost (state):} \$76,273 
%                         \item{20 year ROI:} \$1,350,727
%                     \end{itemize}
%                 \item{Columbia University}
%                     \begin{itemize}
%                         \item{Total [] year cost (state):} \$65,860
%                         \item{20 year ROI:} \$1,361,140 
%                     \end{itemize}
%             \end{itemize}
%         \item{Best Value Schools} 
%             \begin{itemize}
%                 \item{UT Austin}
%                     \begin{itemize}
%                         \item{Total 4 year cost (state):} \$105,384 
%                         \item{20 year ROI:} \$1,321,616 
%                     \end{itemize}
%                 \item{American University}
%                     \begin{itemize}
%                         \item{Total 4 year cost (state):} \$239,552 
%                         \item{20 year ROI:} \$1,187,448 
%                     \end{itemize}
%                 \item{University of California, Los Angeles}
%                     \begin{itemize}
%                         \item{Total [] year cost (state):} \$60,744 
%                         \item{20 year ROI:} \$1,366,256 
%                     \end{itemize}
%                 \item{San Francisco State University}
%                     \begin{itemize}
%                         \item{Total [] year cost (state):} \$38,808 
%                         \item{20 year ROI:} \$1,391,192 
%                     \end{itemize}
%             \end{itemize}
%     \end{itemize}
%     \item{Graduate} 
%     \begin{itemize}
%         \item{Dream Schools}
%             \begin{itemize}
%                 \item{Yale}
%                     \begin{itemize}
%                         \item{3 year cost (state):} \$240,423
%                         \item{20 year ROI:} \$2,058,977
%                     \end{itemize}
%                 \item{Harvard}
%                     \begin{itemize}
%                         \item{3 year cost (state):} \$265,800
%                         \item{20 year ROI:} \$2,033,600
%                     \end{itemize}
%             \end{itemize}
%         \item{Best Value Schools} 
%             \begin{itemize}
%                 \item{University of Alabama}
%                     \begin{itemize}
%                         \item{3 year cost (state):} \$170,688 
%                     \end{itemize}
%                 \item{UT Austin}
%                     \begin{itemize}
%                         \item{3 year cost (state):} \$163,800 
%                         \item{20 year ROI:} \$2,135,600 
%                     \end{itemize}
%             \end{itemize}
%     \end{itemize}
%\end{itemize}

\section{Degree Description}
    \subsection{Undergraduate}
		The Bachelor of Arts program in film and video is for individuals who want to be involved in either cinematography, video editing, or media production. In the curriculum for a BA in film and video, students not only have hands on technical training with film but also study film theory. People who are pursuing a BA in film and video also work on audio production, scriptwriting, film editing, TV interpretation and motion picture production. While earning their BA, students immerse themselves into learning about the film industry by working with other graduates on making films. Any student who is looking into becoming a producer should also look into the communication field. Taking classes in communications can help with marketing, advertising and public relations. Besides taking a variety of classes that help achieve their BA, students should work on extra curricular activities that allow them to use their production skills. Students should work on filming their own videos and help in the production of plays or other theatrical productions.
    \subsection{Graduate}
		While getting a Master’s Degree in film and video isn’t discouraged, it isn’t necessary. Most famous film producers don’t possess a Master's Degree in film and video. The important things for a career in film and video production are internships. Getting your artistic name out into the film industry is far more important than getting another degree. Taking the initiative of getting internships inside and outside of college will lead to better opportunities for jobs. Most production companies don’t hire based off of what school you went to or how many degrees you have, but rather by what you have created in terms of film and video. Having an internship in college is strongly encouraged. For example, if you were to go to school at UT in Austin, Texas, there would be an internship opportunity to work at a production company called Rooster Teeth. There are many different internship opportunities throughout the country, and most colleges set you up with the best internships for you. 

\section{Interview}
	In Steve Prigge’s book \textit{Movie Moguls Speak}, he asked sixteen different film producers, “What do you do as a producer?” Prigge’s goal was to try and understand the different roles that a producer plays in the making of a movie. As Prigge continued with his interviews, he saw that all of the producers agreed that producers don’t get the recognition they deserve. 
	Prigge asked Dino De Laurentiis about his role as a producer. Laurentiis is an Italian film producer who has worked on such movies as \textit{King Kong}, \textit{La Strada} and \textit{Barbarella}. He was one of the first film producers who brought Italian cinema to light. Laurentiis states, 
\begin{quote}
	“The most exciting part for me is the creative side of producing. The producer is, in reality, the soul of the movie. The producer is the one who chooses the script. He chooses the screenwriter. He chooses the director, the cast.” (Laurentiis) 
\end{quote}
	Prigge also asked producer Lauren Shuler Donner. Donner has worked on such films as the \textit{X-Men} movies, \textit{Deadpool}, and \textit{You’ve Got Mail}. Donner specializes in movies for younger audiences and family audiences. Donner states,
\begin{quote}
	“I am there to solve problems . . . come up with an idea to make the film funnier or go faster. At the end of the shoot, I get involved in the cutting room and with the editing, music and video effects, marketing and distribution. To be successful, you have to wear many hats” (Donner).
\end{quote}
	The common theme throughout the interviews is that producers are fundamental to the making of a film. Producers need to be able to do a multitude of jobs and be able to work with different groups of people.

\section{Work-Life Balance}
	Almost always producers are on call for their clients or directors. They are constantly checking their phone email to make sure that the movie isn’t falling apart. Being a producer also means a lot of traveling. It’s hard for a producer to stay in one place at a time, and most producers complain about spending most of their time in airports or cars. In an interview by Susie Schnall, film producer Ria Ruthsatz explains how she handles work and home life balance. Ruthsatz states, “I get really into my work, and can have a really hard time ‘clocking out.’ I’ll sit down to quickly review a budget proposal at 5pm and look up to realize that it’s 2 am!” (Ruthsatz) Ruthsatz goes on to say that as time went on she learned how to better balance her time. Ruthsatz states, 
	
	\begin{quote}
		“In the early days, I was always connected. I wanted to answer whenever the client called, emailed, or texted—day or night. I’ve since learned to designate some time as just plain “off.” I was worried it would impact our business, but in fact it’s allowed me to stay fresh and be more creative and engaged when I am working!” 
	\end{quote}
	
	As for most jobs in the arts, producers feel this need to always be creating and working. With this need, producers need to be able to find a distinction between work life and home life. Most producers are better when they set strict guidelines about when clients and directors can get a hold of them so they aren’t always on call. While the job is incredibly busy and there will be countless late nights, most producers make their own hours and are able to work with other companies to create a work/life balance that works for them.
\end{multicols}