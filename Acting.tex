\chapter{Acting}

Actors tell stories using their voices, their expressions, and their body movements (Occupational Outlook Handbook, Bureau of Labor Statistics, U.S. Department of Labor). Generally speaking, there are two different acting career paths: acting in theatre and acting in film. Actors in theatre will perform on a stage in front of a live audience, while actors in film will perform for a camera and have their scenes edited and replayed. For both acting paths, however, actors must audition for roles, research their characters, interpret scripts, listen attentively to their directors, and rehearse often (U.S. Department of Labor). In order to be successful, actors must be able to read well, speak well, convey emotion, memorize lines, and remain persistent and creative (U.S. Department of Labor). Actors face intense and vast competition for jobs, and they are rarely employed full time (U.S. Department of Labor). Because of this, they often have to seek employment elsewhere in order to support themselves financially. Additionally, actors often have to travel for work and keep long, irregular hours, which may include working nights, weekends, or holidays (U.S. Department of Labor). Comparatively speaking, they don’t earn much money for doing so. In 2015, the median hourly wage for actors was \$18.80, meaning 50\% of working actors earned more, and 50\% earned less (U.S. Department of Labor). The lowest-paid 10\% of working actors earned less than \$9.27 per hour, while the highest-paid 10\% earned more than \$90.00 per hour (U.S. Department of Labor). However, many actors do what they do because they love to act and they feel deeply passionate about it. One cannot put a price on passion. 

\section{Top Colleges}
\begin{itemize}
	\item{Undergraduate} 
	\begin{itemize}
		\item{Dream Schools}
			\begin{itemize}
				\item{University of California - Los Angeles}
					\begin{itemize}
						\item{Total 4 year cost (state):} \$220,000
						\item{20 year ROI:} \$164,000
					\end{itemize}
				\item{Pace University}
					\begin{itemize}
						\item{Total 4 year cost (state):} \$220,000
						\item{20 year ROI:} \$424,000
					\end{itemize}
			\end{itemize}
		\item{Best Value Schools} 
			\begin{itemize}
				\item{University of California - Berkeley}
					\begin{itemize}
						\item{Total 4 year cost (state):} \$225,000
						\item{20 year ROI:} \$443,000
					\end{itemize}
				\item{UT Austin}
					\begin{itemize}
						\item{Total 4 year cost (state):} \$103,000
						\item{20 year ROI:} \$301,000
					\end{itemize}
			\end{itemize}
	\end{itemize}
\end{itemize}

\section{Degree Description}
	\subsection{Undergraduate}
		Majors for actors can be specific to what he or she wants to do, or more broad. For example, a student studying to be an actor could simply major in acting, which would prove beneficial across the board. However, if that student had a particular interest, such as in musical theatre, he or she would want to major in musical theatre, possibly with a minor in dance. Many acting programs include training both for the stage and for film, but some are specific to a particular kind of performance, such as film studies or theatre arts. Regardless of their career path, actors will most likely need to take classes that involve acting, studying and interpreting scripts, learning how to use their voice, learning how to move their bodies, and actual production-based classes. Any student aspiring to be an actor should not only look into acting and drama programs and classes at their school, but should also try to participate in as many extracurricular productions as possible, whether they’re staged or filmed. What an actor chooses to major in does not matter nearly as much as the time they spend working on and auditioning for shows, gaining experience, building a resume, and making connections.
	\subsection{Graduate}
		Many actors decide to get to work straight out of college, without going to graduate school. Doing so certainly has its benefits, particularly since the sooner one can find an agent and get into the business, the better. However, actors can find many benefits to pursuing an MFA (master of fine arts). Graduate school classes tend to involve the same subjects as undergraduate classes, but can be more specialized and more professional. Within those extra two or three years of study, actors can gain credibility, become more polished and professional, establish strong connections with those in the field and make new contacts, broaden their amount of experience, and strengthen their resume (Brian O’Neil, “7 Reasons Why an Actor Gets an MFA”). Additionally, according to former agent and current acting career coach, Brian O’Neil, “A number of top programs are tuition-free, or close to it, and some offer a living stipend as well. Many actors in MFA programs are actually spending less money on continued training than their peers who are studying elsewhere” (“7 Reasons Why an Actor Gets an MFA”). So, there can even be financial benefits to taking part in a graduate program. All in all, the benefits of graduate school really pertain to the connections, training, experience, and resume of the actor. If he or she can find an agent and relatively regular work shortly after college, it would make sense to avoid spending extra money on graduate school. However, those who have the opportunity and resources to go to a good graduate school program can benefit immensely from doing so. 

\section{Interview}
	Calista Hernandez did not know she wanted to be an actor at the age of 16, or 18, or even when she graduated from college. During her college years, she changed her major about every semester, moving from music composition to art, to interior design, to journalism, to advertising, and then back to journalism. She attended a few different schools, eventually graduating from UT with a major in photojournalism, but she still felt unsure of what she wanted to do with her life. For Calista, acting did not prove to be so much of a decision as it did a revelation. In a recent interview, when asked about why she decided to become an actor, she informed me,  

	She does wish she could have gone to a drama school, and for a while, she felt embarrassed that she did not possess any sort of educational background in acting. However, she met some incredible teachers along the way, and ultimately, her life experiences taught her what she needed to learn. She learned how to take the various criticisms thrown at her. She learned how to accept that a lot of work opportunities depended on her luck. And she learned how to be both tough and vulnerable at the same time. When I asked her what skills she felt were helpful to working actors, she advised,
	\begin{quote}
		“To be a working actor? Tough skin, luck and confidence. To be a good working actor? All that and willingness . . . Trust yourself. Read a lot. Watch a lot of films. Have strong boundaries. Listen deeply. Keep going. Tell the truth. Trust your intuition. Go on a tangent- learn about everything that interests you . . . It's only your life, you might as well try, right?”
	\end{quote}
	\begin{quote}
		“I think with any creative endeavor, it's less of a choice and more of a knowing . . . I may not have known consciously when I "decided," but, in short, telling stories, tapping into a collective unconscious, digger deeper in your own unconscious self and the incessant feelings and understanding of the human heart. Connection. I hope this doesn't sound too pretentious! It's really got nothing to do with hours, money, travel, fame or any of that, for me. It's extremely personal. And it's extremely vulnerable. I love that. ”
	\end{quote}

\section{Work-Life Balance}
		There’s a lot more to the acting process than meets the eye. Firstly, those who cannot support themselves with their acting salary alone must also look for other work and hold another job. However, even for working actors, they cannot spend all their time doing what they’re most passionate about.They have to tend to all of the other work that goes along with a career in acting. There’s preparing for auditions, actually auditioning, seemingly endless traveling, studying and analyzing characters and scripts, not to mention all the time spent in hair and makeup and just waiting around in the trailer. Sometimes a whole day is spent just waiting. The time spent actually acting and shooting a scene is often outweighed by all the extra work surrounding the acting. However, the most difficult part of acting does not involve any of these necessary extra tasks. For Calista, the most challenging aspect of an acting career is the sheer sense of isolation that comes with it. When describing her career, Calista explained,
		\begin{quote}
			“It's an incredibly lonely profession. I don't think people realize that . . . You go to a remote place, far away from anyone or anything you know . . . You get to see and meet 200 new faces which you'll work closely with for the next month or two or three. You go to work . . . Pressure and nerves sometimes get the best of you. Sometimes not. You may have a few days or even a week off in this remote town that you cannot leave. It takes finessing to find happiness in that.”
		\end{quote}
		One would probably assume that getting to meet 200 new people means tons of new friends to talk with, but the time spent working with these people only lasts a few months. Often, they’re all in their own world or their own trailer. Actors have to work to entertain themselves, finding new things to do on their own in a place they’ve never been before. The sense of detachment and solitude can prove draining. When I asked Calista about the balance between work and life, she replied,
		\begin{quote}
			“It's difficult. I'm learning as I go that, although these two things are extraordinarily intertwined, they are also separate. It's tricky because each one informs the other. Your life and your person informs your work. Work informs your life and your person.”
		\end{quote}
		She’s still searching for that healthy balance. However, many people spend their entire lives searching for such a balance. Few have it all figured out by age 29. At least Calista finds magic in what she does. She found what she needed to do, and she hasn’t looked back. 