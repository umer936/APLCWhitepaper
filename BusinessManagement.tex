\chapter{Business Management}
\begin{multicols}{2}

Not many people can deny that business makes the world go ‘round. Everyone is either buying, selling, or producing goods to be sold. But the masterminds of this fast-paced world are the corporate managers and CEOs of major companies. These are the people who understand how to apply successful business techniques and strategies to enable any business to succeed in any industry (“Business Majors: The Basics”). Some of the skills needed to be able to successfully manage a business include the ability to plan, organize, make decisions, communicate, and lead in a variety of work settings.  According to Payscale, the median salary for someone with a degree in business management is around \$62,500 (“Best Value Business Schools” ). However, since the business world is full of busts and booms due to the economy, one can go from hitting it big and making millions and even billions to bankrupt with the snap of a finger. Luck, availability of resources, and networking play a major role in the success of anyone aspiring to enter into the world of business. 

\end{multicols}

\section{Top Colleges}

\begin{table}[H]
\centering
\caption{Undergraduate Colleges}
\label{Buisness Management Undergraduate Colleges}
\resizebox{\textwidth}{!}{%
\begin{tabular}{llrlr}
\hline
\multicolumn{5}{|l|}{Dream Schools}                                                                 \\ \hline
Massachusetts Institute of Technology & Total 6 year cost: & \$280,224 & 20 year ROI: & \$1,270,776 \\
University of Pennsylvania            & Total 6 year cost: & \$297,216 & 20 year ROI: & \$908,784   \\ \hline
\multicolumn{5}{|l|}{Best Value Schools}                                                            \\ \hline
SUNY Maritime College                 & Total 4 year cost: & \$126,000 & 20 year ROI: & \$945,000   \\
Babson College                        & Total 4 year cost: & \$238,000 & 20 year ROI: & \$794,500  
\end{tabular}%
}
\end{table}

\begin{table}[H]
\centering
\caption{Graduate Universities}
\label{Buisness Management Graduate Universities}
\resizebox{\textwidth}{!}{%
\begin{tabular}{llrlr}
\hline
\multicolumn{5}{|l|}{Dream Schools}                                              \\ \hline
Stanford                 & Yearly cost: & \$126,600 & 20 year ROI: & \$736,200   \\
Harvard                  & Yearly cost: & \$139,186 & 20 year ROI: & \$751,558   \\ \hline
\multicolumn{5}{|l|}{Best Value Schools}                                         \\ \hline
Brigham Young University & Yearly cost: & \$49,249  & 20 year ROI: & \$1,890,760 \\
Indiana University       & Yearly cost: & \$116,790 & 20 year ROI: & \$1,916,330
\end{tabular}%
}
\end{table}

\begin{multicols}{2}
% \begin{itemize}
%     \item{Undergraduate} 
%     \begin{itemize}
%         \item{Dream Schools}
%             \begin{itemize}
%                 \item{Massachusetts Institute of Technology}
%                     \begin{itemize}
%                         \item{Total 6 year cost (state):} \$280,224 
%                         \item{20 year ROI:} \$1,270,776
%                     \end{itemize}
%                 \item{University of Pennsylvania}
%                     \begin{itemize}
%                         \item{Total 6 year cost (state):} \$297,216 
%                         \item{20 year ROI:} \$908,784
%                     \end{itemize}
%             \end{itemize}
%         \item{Best Value Schools} 
%             \begin{itemize}
%                 \item{SUNY Maritime College}
%                     \begin{itemize}
%                         \item{Total 4 year cost (state):} \$126,000 
%                         \item{20 year ROI:} \$945,000
%                     \end{itemize}
%                 \item{Babson College}
%                     \begin{itemize}
%                         \item{Total 4 year cost (state):} \$238,000
%                         \item{20 year ROI:} \$794,500
%                     \end{itemize}
%             \end{itemize}
%     \end{itemize}
%     \item{Graduate} 
%     \begin{itemize}
%         \item{Dream Schools}
%             \begin{itemize}
%                 \item{Stanford}
%                     \begin{itemize}
%                         \item{Yearly cost (state):} \$126,600
%                         \item{20 year ROI:} \$736,200
%                     \end{itemize}
%                 \item{Harvard}
%                     \begin{itemize}
%                         \item{Yearly cost (state):} \$139,186
%                         \item{20 year ROI:} \$751,558
%                     \end{itemize}
%             \end{itemize}
%         \item{Best Value Schools} 
%             \begin{itemize}
%                 \item{Brigham Young University}
%                     \begin{itemize}
%                         \item{Yearly cost (state):} \$49,240
%                         \item{20 year ROI:} \$1,890,760
%                     \end{itemize}
%                 \item{Indiana University}
%                     \begin{itemize}
%                         \item{Yearly cost (state):} \$116,790
%                         \item{20 year ROI:} \$1,916,330
%                     \end{itemize}
%             \end{itemize}
%     \end{itemize}
% \end{itemize}

\section{Degree Description}
    Business management is a field that offers unlimited opportunities for individuals with an interest in business and technology who work well with all people and who have excellent communication skills (“Careers In Management”). To obtain a Business Management Degree from most colleges, students must complete classes pertaining to human resources, negotiations, microeconomics, and data analysis (2015-2016 Undergraduate Business School Handbook.). However, since there are a variety of careers associated with Business Management, colleges usually offer “tracks” (“Careers In Management”). These tracks help you obtain a degree in Business Management but allow students to specialize in a certain area. These tracks often include general management, entrepreneurial leadership, human resource management, nonprofit management, or Pre-Law (2015-2016 Undergraduate Business School Handbook). However, students need to keep in mind when choosing a track that the basic business management courses are still the same (2015-2016 Undergraduate…). After these courses are completed, the required classes for each student will change based on the track that they choose (2015-2016 Undergraduate…). Now, because internships are not usually required, it is strongly recommended that students seeking to obtain a degree in business management join a student organization such as: the Business Administration Society, Entrepreneurship Society, Society for Human Resource Management, or the Business Student Council (“Careers in Management”).. In fact, some schools push for students to join these groups more often than they encourage students to intern, because of the connections that these groups offer and it aids a business management major because the student is forced to work with and interact with others with similar interests and more closely emulates the work setting of a manager than does an internship (Orsak). Furthermore, students that seek to obtain a Masters degree in Business Management would like to further develop their knowledge of the business world so just like anyone wishing to obtain a bachelors in Business Management, they would have to choose a track and take the required classes (“Careers in Management”). However, the student would not be responsible for the prerequisite classes required of undergraduates (“Careers in Management”). And although internships are not required, it is recommended that the student take full advantage of school-sponsored clubs and available internships (“Careers in Management”).
\section{Interview}
I chose to interview two different people; one of which is currently running and managing his own business and one who is currently studying Business Management in college.
    Dale Crenwelge attained his Masters of Business while attending Texas A\&M University in College Station, TX. He is “presently self employed as President of fifteen different real estate ventures located in Texas, Florida, Georgia, and South Carolina” (Crenwelge).  He was previously employed as president of Crenwelge Commercial Consultants, Inc. and worked as project engineer with ConocoPhillips in Oklahoma. He has served on numerous school boards and currently serves on the board of Capital Farm Credit, Trustee of the Hill Country Memorial Hospital in Fredericksburg, and the Kendall County Centurions in Boerne (Crenwelge). When asked about the best part of his job, Mr Crenwelge stated that seeing his customers satisfied with his work was one of the most rewarding aspects of his job (Crenwelge). However, he also commented that it is equally rewarding to resolve the daily challenges and conflicts that arise among many of the people he supervises (Crenwelge). In closing, Mr. Crenwelge left me with this advice, “first, do something you enjoy and devote all of your energies to it; don’t ever give up. And second, integrity is essential to success; so let everything you do be done with integrity” (Crenwelge).
    Sarah Orsak is a student at the University of Texas A\&M in College Station, TX and is majoring in Business Management. Sarah states that one of her main reasons for wanting to attain a degree in business management “is because of the wide range of opportunities it provides for the future” she then continued to elaborate on how she believed that the degree would provide her with the ability to work at almost any company and would give her the skills and knowledge necessary to succeed in starting her own business (Orsak). Although Sarah is not planning on obtaining work experience or participating in internships prior to her graduation, she says that she is apart of many different clubs on campus that relate to her major and that these clubs have given her a large network of people with similar interests to communicate and brainstorm with (Orsak). Sarah says that the best part about being a business management major at A\&M is that you are only set up for success. She elaborates on the various help centers specific to the business school that provide “aid in everything from setting up a resume to interview skills, to absolutely anything that will be an asset once you enter the business world’ (Orsak). Sarah’s closing advice for any aspiring Business Management major is to “be flexible and don’t be afraid to try new things, get involved into groups related to the part of business you currently plan on entering but don’t be afraid to alter that path if you start to envision a different future for yourself” (Orsak).
    Although people who major in business management tend to make more money, it comes at a price. People who pursue careers in Business Management often face around a 60-hour work week which is usually spent traveling. (“Management Majors Guide.”) The difficulty of a career in the business world is echoed in both of the interviews previously mentioned. Mr. Crenwelge pointed out that the hardest part of his job was balancing his combination of a strong work ethic and long hours while maintaining a connection with his family (Crenwelge). Sarah’s description of the large time commitment required to succeed in school reflects the type of work week she will face once she graduates (Orsak). Because Sarah is a varsity athlete at A\&M, she says that it is extremely hard to balance the heavy workload handed out by the school with her athletic training and still manage to have some sort of a social life (Orsak). However, both Sarah and Mr. Crenwelge emphasized the importance of planning for each day to make sure that all time is accounted for and none is wasted. Both also express that they feel most accomplished when they overcome challenge that seemed nearly impossible to solve upon first glance. In conclusion, if you are not the type of person that enjoys a challenge, then Business Management probably isn’t the career for you. However, if you are someone who plans to dedicate themselves wholeheartedly to their work, then I think you have found your calling. 
\end{multicols}