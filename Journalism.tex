\chapter{Journalism}
\begin{multicols}{2}

Who? What? When? Where? Why? These are the questions that journalists aim to answer. While the age of newspaper is fading away, the electronic age is quickly expanding. However, the lives of most journalists are not luxurious or expensive. Depending on the specific field, a typical day for a journalist might involve conducting interviews, writing and editing stories on a deadline, taking photos, working long hours, going to news conferences, and reporting in front of a camera (“Journalist Job Description”). To achieve these tasks, a journalist needs to be able to communicate well both in writing and in verbal conversation, to not buckle under pressure, and to cooperate in groups (“Journalist Job Description”). PayScale conducted a survey of 660 journalists, and the results show that the annual salaries ranged from \$24,151 to \$71,166 (“Journalist Salary (United States)”). Additionally, the Bureau of Labor Statistics noted that 54,000 reporters, correspondents, and broadcast news analysts with bachelor degrees had a median pay of \$37,200 per year in 2014 (“Occupational Outlook Handbook”). The salary of a journalist is affected by the amount of time he or she has been working; on average, the longer a journalist works, the more money he or she will earn (“Journalist Salary (United States)”). To earn the most money, journalists should live in Washington, D.C., New York, San Francisco, San Diego, or Chicago (“Journalist Salary (United States)”). Monetarily, the most profitable position within journalism is the senior editor (“Sr. Editor Salary (United States)”). Most senior editors are females who have more than 10 years of experience, and they receive an average salary of \$62,000 per year (“Sr. Editor Salary (United States)”).  

\end{multicols}

\section{Top Colleges}

\begin{table}[H]
\centering
\caption{Undergraduate Colleges}
\label{Journalism Undergraduate Colleges}
\resizebox{\textwidth}{!}{%
\begin{tabular}{llrlr}
\hline
\multicolumn{5}{|l|}{Dream Schools}                                                           \\ \hline
University of Southern California & Total 4 year cost: & \$278,844 & 20 year ROI: & \$465,156 \\
Northwestern University           & Total 4 year cost: & \$272,348 & 20 year ROI: & \$471,652 \\ \hline
\multicolumn{5}{|l|}{Best Value Schools}                                                      \\ \hline
American University               & Total 4 year cost: & \$172,408 & 20 year ROI: & \$571,591 \\
UT Austin                         & Total 4 year cost: & \$85,640  & 20 year ROI: & \$658,360
\end{tabular}%
}
\end{table}
\begin{table}[H]
\centering
\caption{Graduate Universities}
\label{Journalism Graduate Universities}
\resizebox{\textwidth}{!}{%
\begin{tabular}{llrlr}
\hline
\multicolumn{5}{|l|}{Dream Schools}                                                            \\ \hline
Columbia University                & Total 4 year cost: & \$122,242 & 20 year ROI: & \$233,758 \\
University of California, Berkeley & Total 4 year cost: & \$74,203  & 20 year ROI: & \$281,797 \\ \hline
\multicolumn{5}{|l|}{Best Value Schools}                                                       \\ \hline
Arizona State University           & Total 4 year cost: & \$53,472  & 20 year ROI: & \$302,472 \\
UT Austin                          & Total 4 year cost: & \$22,324  & 20 year ROI: & \$333,676
\end{tabular}%
}
\end{table}

\begin{multicols}{2}
% \begin{itemize}
%     \item{Undergraduate} 
%     \begin{itemize}
%         \item{Dream Schools}
%             \begin{itemize}
%                 \item{University of Southern California}
%                     \begin{itemize}
%                         \item{Total 4 year cost (state):} \$278,844
%                         \item{20 year ROI:} \$465,156
%                     \end{itemize}
%                 \item{Northwestern University}
%                     \begin{itemize}
%                         \item{Total 4 year cost (state):} \$272,348
%                         \item{20 year ROI:} \$471,652
%                     \end{itemize}
%             \end{itemize}
%         \item{Best Value Schools} 
%             \begin{itemize}
%                 \item{American University}
%                     \begin{itemize}
%                         \item{Total 4 year cost (state):} \$172,408
%                         \item{20 year ROI:} \$571,592
%                     \end{itemize}
%                 \item{UT Austin}
%                     \begin{itemize}
%                         \item{Total 4 year cost (state):} \$85,640
%                         \item{20 year ROI:} \$658,360
%                     \end{itemize}
%             \end{itemize}
%     \end{itemize}
%     \item{Graduate} 
%     \begin{itemize}
%         \item{Dream Schools}
%             \begin{itemize}
%                 \item{Columbia University}
%                     \begin{itemize}
%                         \item{Total 4 year cost (state):} \$122,242
%                         \item{20 year ROI:} \$233,758
%                     \end{itemize}
%                 \item{University of California at Berkeley}
%                     \begin{itemize}
%                         \item{Total 4 year cost (state):} \$74,203
%                         \item{20 year ROI:} \$281,797
%                     \end{itemize}
%             \end{itemize}
%         \item{Best Value Schools} 
%             \begin{itemize}
%                 \item{Arizona State University}
%                     \begin{itemize}
%                         \item{Total 4 year cost (state):} \$53,472
%                         \item{20 year ROI:} \$302,528
%                     \end{itemize}
%                 \item{UT Austin}
%                     \begin{itemize}
%                         \item{Total 4 year cost (state):} \$22,324
%                         \item{20 year ROI:} \$333,676
%                     \end{itemize}
%             \end{itemize}
%     \end{itemize}
% \end{itemize}

\section{Degree Description}
    \subsection{Undergraduate}
Journalism is unique because of the numerous degrees and subjects within it. According to the “Major and Career Search” on the College Board website, there are approximately 17 different sectors within the broad journalism topic, including broadcast, mass communications, scientific and technical communication, sports communication, photojournalism, and digital communications and multimedia. Jenna Goudreau, a writer at Forbes, said in an article that although many journalists are successful without journalism degrees, having a journalism degree would be helpful. She also recommended double majoring, earning a degree in journalism and a degree in a completely different subject (Goudreau). According to Justin Cox, who attended graduate school at Northwestern, having a graduate degree in journalism is the only way to have a job above entry level positions. However, graduate school is expensive, and journalists do not get paid very much, as opposed to doctors, who also have expensive educations but are able to pay off student loans easier (Cox). 
\section{Interview}
    In a recent interview, Chuck Miketinac, a local sportscaster, talked about his career as a journalist. He has been working at KABB, FOX San Antonio for 20 years. When he was in high school, he decided that because he loved sports and enjoyed writing, he would be a sports journalist. At that point, he did not know that he would end up being on TV. However, when he was in college, he did want to be on TV, and so he took classes based on that desire. While Miketinac was interning at KSAT, one of his friends that was leaving the job recommended that Miketinac be hired—and he was. When asked about his work-life balance, Miketinac said, “I'm sure I'd like to work less, but who doesn't want that?” Miketinac went on to say that he does not bring any work home. According to Miketinac, the most difficult aspect of his job is staying up-to-date so that he can constantly report. To conclude, he gave the following advice: “If you want to do this for a career, don't let anybody tell you that you can't do it. It's too hard, too competitive is a bunch of bull” (Miketinac).
\section{Work-Life Balance}
    In a Poynter survey, 750 journalists answered questions about their jobs. Although most of these people reported high job satisfaction, almost 50\% of the participants have considered leaving their careers due to work-life balance issues, such as extensive work hours and the loss of vacation time (Geisler). However, less than half of the employees asked their supervisors for short-term improvements; of those who did ask, almost 75\% received some aid (Geisler). More than 55\% of the respondents told that their organizations demonstrated at least some concern for the work-life balance of their employees (Geisler). 51.6\% of the participants in the survey reported that their supervisors were unsupportive, while 48.4\% said that they had supportive supervisors (Geisler). In addition, 61.8\% of the responders answered that their jobs have a somewhat negative effect on their physical health (Geisler). 
\end{multicols}