\chapter{JD-MBA}

A JD-MBA degree is a joint degree that offers college students the opportunity to graduate with a master’s degree in both business and law. Before this degree existed, many college students who wanted to have degrees in both law and business had to attend college for about 9 years. However, this is not the case anymore. Thanks to this new program, students are now able to graduate in about 7-8 years (including undergraduate). This accelerated course provides immense help to students attending prestigious colleges, such as Harvard or Yale, because it saves them an immense amount of time and money. Also, they will be able to begin work earlier, and this is very important because many of these students have enormous student loans that they must pay.  Some universities, such as Columbia and Yale, offer 3-year accelerated programs that allow a student to jump into the working world as quick as possible. (O’Connor, “5 Benefits of a JD/MBA”). A JD-MBA degree is very helpful for students who not only desire to go into the business world, but also want a backup job in case their business goes south. Being a lawyer helps out with your career safety a lot. Potential careers with this degree range from Hedge-fund manager, stock broker, CEO/CFO, lawyer, attorney, judge, partner in firm, and many more jobs. Honestly, a JD-MBA graduate has the ability to hold any position within either the business or law sector. In other words, you can feel pretty confident that you will be making serious money just years after you graduate. Let's say that upon graduating, you are offered a job at a terrible law firm. This law firm is going to pay you 90,000 dollars per year (which is extremely low), and you will have this set salary for twenty years (this is practically the worst scenario possible). Within the first 12 years, your net earnings will be over a million dollars; however, we all know that there will be expenses so we can reduce this number to somewhere around half a million dollars. Either way, after 12 years of working in the worst possible conditions for this field, you are halfway there to being a millionaire. So, as you can see, the ROI for this degree is immense, even if one finds him or herself in this terrible situation.  

\section{Top Colleges}
\begin{itemize}
    \item{Graduate} 
    \begin{itemize}
        \item{Dream Schools}
            \begin{itemize}
                \item{Yale University}
                    \begin{itemize}
                        \item{4 year cost (state):} \$225,056 
                        \item{20 year ROI:} \$2,974,944 
                    \end{itemize}
                \item{Stanford}
                    \begin{itemize}
                        \item{4 year cost (state):} \$225,056
                        \item{20 year ROI:} \$2,974,944 
                    \end{itemize}
            \end{itemize}
        \item{Best Value Schools} 
            \begin{itemize}
                \item{UT Austin}
                    \begin{itemize}
                        \item{4 year cost (state):} \$39,320 
                        \item{20 year ROI:} \$3,160,680 
                    \end{itemize}
                \item{University of Alabama}
                    \begin{itemize}
                        \item{4 year cost (state):} \$60,000 
                        \item{20 year ROI:} \$3,140,000 
                    \end{itemize}
            \end{itemize}
    \end{itemize}
\end{itemize}

\section{Degree Description}
	Graduates with JD-MBA degrees often times find much more success in the businesses world than other people in their university. Starting salaries for JD-MBA graduates are extremely high, with many graduates having a first year salary starting at an average of 160,000 dollars. Also, most companies give JD-MBA graduates bonuses upon signing with them, usually these bonuses start at 20,000 dollars. However, it is important to note that many financial or legal firms require graduates to have at least one year of working experience before they will be given a contract such as the one above. Also, many students will intern at companies before they graduate, and many times if the company likes the student; they will offer him or her a contract that offers to pay one year of their college tuition/expenses along with a huge starting salary and a bonus. (Abraham, “Pros and Cons of the JD/MBA.”). Another thing that is great for JD-MBA graduates is that they are almost always promoted to positions of power before their other co-workers, and this is because almost all of them have not only extensive business knowledge, but also a profound understanding of law. Much like the basic economic principle of supply and demand, JD-MBA graduates are rare, and because of this their value to companies rises exponentially each year.  Due to this, companies and corporations fight to try and sign the best JD-MBA graduates each year. Graduates who have JD-MBA degrees often times find great success in the work world. They have great flexibility when it comes to choosing careers. These graduates can change their career with ease, ranging anywhere from lawyers to financial traders, or vice-versa. A testimony to this success is represented in the following statistic: “In fact,  of Fortune 500 CEO’s, 46 hold JDs.”( O’connor, “5 Benefits of a JD/MBA”). When it comes to getting bonuses, partnership, or raises, JD-MBA graduates seem to excel due to their great understanding of how the business works, along with their profound knowledge of the laws that hold the business in place.

\section{Interview}
	So my interview involves a former college student and a website that helps students apply to college, and the name of this website is Accepted. Accepted interviewed a former student who attained a JD-MBA degree at Wharton University. This student’s name is Craig Carter. Craig is an African-American student who received great grades at Wharton, so he was interviewed in the interest of seeing how he was coping with the difficulty of the JD-MBA degree. Throughout this interview, Craig makes it clear that he has more than enough time to do other activities such as hang out with friends or be active in school-clubs. (Craig Carter, Interview). Craig also really emphasized on his appreciation for the immense amount of networking that he made at Wharton. He also makes it clear to us that the transition between business and law is very interesting, but that he personally took a much deeper liking to business. (Craig Carter, Interview). This is something that is fascinating because he demonstrates the fact that eventually each student will tend to enjoy one subject more than the other. Craig then goes on to explain how this degree, along with his school, has helped him attain an internship at the exclusive JP Morgan group in New York.(Craig Carter, Interview). This functions as a testimony to the exclusive opportunities that a JD-MBA degree can provide for its attainers. Overall, Craig lets the reader know that he is very happy with his degree, and he can't wait to jump into the business world. 

\section{Work-Life Balance}
	Students who do study JD-MBA degrees often struggle during their education because switching from very hard law to very hard math is tricky for many students. Because of this [struggle], many students often times are unhappy while they are studying. However, upon graduation, a large number of these students are contracted by powerful firms that richly recompense them for their work, so if a person is willing to stick it out for three years, then ultimately this degree will be more than worth it. Once the students graduate, the majority will join exclusive law firms or join some sort of financial investment association. Here, they will work long hours, and they will have very busy schedules. However, they will be making money, and as they rise in the company, their work days become shorter, but more action packed. One must really like what he or she will be doing, or else this career can be very stressing and unfulfilling. 
