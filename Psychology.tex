\chapter{Industrial and Organizational Psychology}

Industrial and Organizational Psychology is the scientific study of human behavior in the workplace in order to improve work-life balance and efficiency using psychological theories. I-O’s spend their time working for businesses, studying the workplace, and ensuring that the employees are as happy, healthy, and efficient as possible through consultation. I-O’s find themselves working one on one with employees, in research groups, and with the CEOs (Jack Kelle). However, they will never find themselves working alone. I-Os do a lot of research, but not in isolation. It is often done out in the work field. I-O’s must get a feel for the leaders they are working with by interacting with their employees in order to make an accurate analysis. Job analysis of the employees is extremely important as well as an ability to coach the employees (Alissa Parr, Ph.D). I-O’s are considered one of the most stable occupations out of the psychology occupations. Businesses have been known to hire psychologists right out of school. The mean income is \$151,135 (calculated by averaging the highest and lowest income). The lowest that an I-O can make a year is \$52,270, while the highest is \$250,000 and above (“Occupational Employment and Wages, May 2015: 19-3032 Industrial-Organizational Psychology”). 

\section{Top Colleges}
\begin{itemize}
    \item{Undergraduate} 
    \begin{itemize}
        \item{Dream Schools}
            \begin{itemize}
                \item{Harvard}
                    \begin{itemize}
                        \item{Total 4 year cost (state):} \$276,000 
                        \item{20 year ROI:} \$2,746,700
                    \end{itemize}
                \item{Pennsylvania State University}
                    \begin{itemize}
                        \item{Total 4 year cost (state):} \$186,908 
                        \item{20 year ROI:} \$2,835,792 
                    \end{itemize}
            \end{itemize}
        \item{Best Value Schools} 
            \begin{itemize}
                \item{The University of Wisconsin}
                    \begin{itemize}
                        \item{Total 4 year cost (state):} \$100,764 
                        \item{20 year ROI:} \$2,921,936 
                    \end{itemize}
                \item{University of California, Berkeley}
                    \begin{itemize}
                        \item{Total 4 year cost (state):} \$142,968 
                        \item{20 year ROI:} \$2,879,732
                    \end{itemize}
            \end{itemize}
    \end{itemize}
    \item{Graduate} 
    \begin{itemize}
        \item{Dream Schools}
            \begin{itemize}
                \item{University of Minnesota}
                    \begin{itemize}
                        \item{6 year cost (state):} \$118,830 
                        \item{20 year ROI:} \$2,903,870 
                    \end{itemize}
                \item{Michigan State}
                    \begin{itemize}
                        \item{6 year cost (state):} \$118,830
                        \item{20 year ROI:} \$2,903,870
                    \end{itemize}
            \end{itemize}
        \item{Best Value Schools} 
            \begin{itemize}
                \item{Bowling Green State University}
                    \begin{itemize}
                        \item{6 year cost (state):} \$108,174 
                        \item{20 year ROI:} \$2,914,526 
                    \end{itemize}
                \item{University of Southern Florida}
                    \begin{itemize}
                        \item{6 year cost (state):} \$103,950 
                        \item{20 year ROI:} \$2,918,750 
                    \end{itemize}
            \end{itemize}
    \end{itemize}
\end{itemize}

\section{Degree Description}
    \subsection{Undergraduate}
        Courses in sociology, statistics, political science, science, math, research methods, business and psychology are crucial, if not important, for I-O psychologists. (“Organizational Psychology Degrees: What You’ll Study”). Sociology studies the organizations of societies, which includes the causes for the changes within society and the relationships within them (“Sociology Graduate Programs \& Sociology Grad Schools”). Statistics deals with the interpretations and classifications of data. In political science, you will analyze the science of dealing with political institutions of government (“Statistics Graduate Programs \& Data Analytics Graduate Programs”). Business Psychology is the study and practice of improving the work-life relationship. It is also beneficial to take a course in Organizational Management (“Business Psychologist: Career Info, Job Duties, and Requirements”). There is no required internship for Industrial and Organizational Psychology, however, it is beneficial. 
    \subsection{Graduate}
        Business psychologists apply research and research to study people, workplaces, and organizations. The end goal is to identify what creates the best work-life relationship (“Business Psychologist”). The graduate program of Organizational Development studies, researches, and theorizes organizational change and performance. As a degree, Organizational Leadership allows I-Os to determine how efficient a company can be by identifying the goals of the company and the structural problems they may have (“Organizational Leadership Graduate Programs \& Organizational Leadership Graduate Schools”).

\section{Interview}
    Rachel Zoe worked as a Personnel Research Psychologist for the GS 12-13 services in San Antonio for the Air Force. During her time as an Industrial and Organizational Psychologist, Zoe was able to travel the world and visit places including Australia. She majored in psychology and considers herself as a moderately successful in the field. When I asked her what her overall work-life balance was for her job, she said,
    \begin{quote}
        “I did not work too long of hours and I have a family of four so I am still able to manage  that. My managing skills from my job can still sometimes bleed into my home-life.”
    \end{quote}
    I then asked her to describe a day in her shoes as an I-O; 
    \begin{quote}
        “Daily duties include making decisions to benefit the company’s health. Designing plans  is also a large part of my job. I design strategies for my company to have a high success  rate and create leaders. I also work with a team of managers to analyze data of the company, create employee surveys, and create goal-setting plans.”
    \end{quote}
    She then explained what her overall experiences with grad and undergrad school,  
    \begin{quote}
        “At the time when I was in undergrad school, college was completely different. I ended  up majoring in psychology and was able to do well in my field. My leadership skills really came in handy.”
    \end{quote}
    After receiving insight on her undergraduate and graduate experience, I asked her what the most rewarding thing was about her job? She stated that 
    \begin{quote}
        “Although my job can affect my psyche and I sometimes end the day beyond fatigued, it   is worth it. Watching people and businesses transform before my own eyes is the most rewarding thing. Seeing people open up and let their true selves out is the most rewarding.” 
    \end{quote}
    I questioned what she would warn someone of before they enter the field of psychology and she said, 
    \begin{quote}
        “What would you warn someone of before they enter psychology? If you want to study   psychology, get ready to commit. You must be extremely patient and ready for challenging cases. Be ready to not be rewarded for every single little thing you do. Your patients will improve—the majority, if not all, but you have to be patient and make sure to remember that even if you feel like the bad guy, you are not.”
    \end{quote}

\section{Work-Life Balance}
Psychologists in this career often set their own hours. Psychologists in private practice create their own hours, however they often can find themselves working evening or weekend shifts if they are employed by a healthcare career. Those working in clinics, government facilities, industries, or schools will work according to the standard business hours. (“Industrial-Organizational Psychologists: Overall Kind of Work”). 
