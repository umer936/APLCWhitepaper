\chapter{Engineering}
\begin{multicols}{2}

Engineering is “the branch of science and technology concerned with the design, building, and use of engines, machines, and structures.” Being an engineer requires skills like creativity, being able to plan a project in regards to time and money, being good at understanding higher level math, and being able to use specialized tools and equipment (Snagajob). It is a very large field consisting of over ten specializations, which include everything from mechanical engineering to computer science. According to the Bureau of Labor Statistics in 2010, Computer Engineers’ median pay is \$98,810 per year while Agricultural Engineers’ median pay is \$71,090 per year (Best Value Schools). However, the disparity in specializations is even higher, with petroleum engineering’s median pay at \$114,080 per year. In order to become an engineer, one must obtain at least one bachelor’s degree in an engineering field (Snagajob). However, some engineers get multiple bachelor’s degrees and others get master’s degrees (Snagajob). Personally, I am going into Electrical and Computer Engineering, though I love Mechanical Engineering and Aerospace Engineering as well. Thus, these are the 3 fields I will use for my information. Electrical and Computer Engineering consists of designing and creating technology for things like “the automobile, radio and television, computers, spacecraft and the Internet” (“Electrical and Computer Engineering”). Mechanical Engineering, on the other hand, deals more with the physical aspects, like machinery and robotics (“Mechanical Engineering”). 

\end{multicols}

\section{Top Colleges}

\begin{table}[H]
\centering
\caption{Undergraduate Colleges}
\label{Engineering Undergraduate Colleges}
\resizebox{\textwidth}{!}{%
\begin{tabular}{llrlr}
\hline
\multicolumn{5}{|l|}{Dream Schools}                                                               \\ \hline
Massachusetts Institute of Technology & Total 4 year cost: & \$232,000 & 20 year ROI: & \$962,000 \\
Stanford                              & Total 4 year cost: & \$240,000 & 20 year ROI: & \$854,000 \\ \hline
\multicolumn{5}{|l|}{Best Value Schools}                                                          \\ \hline
UT Austin                             & Total 4 year cost: & \$86,000  & 20 year ROI: & \$931,100 \\
Georgia Tech                          & Total 4 year cost: & \$210,000 & 20 year ROI: & \$756,000
\end{tabular}%
}
\end{table}

\begin{table}[H]
\centering
\caption{Graduate Universities}
\label{Engineering Graduate Universities}
\resizebox{\textwidth}{!}{%
\begin{tabular}{llrlr}
\hline
\multicolumn{5}{|l|}{Dream Schools}                                                  \\ \hline
Massachusetts Institute of Technology & Yearly cost: & \$46,000 & 20 year ROI: & \$\$\$ \\
Stanford                              & Yearly cost: & \$48,720 & 20 year ROI: & \$\$\$ \\ \hline
\multicolumn{5}{|l|}{Best Value Schools}                                             \\ \hline
UT Austin                             & Yearly cost: & \$9,564  & 20 year ROI: & \$\$\$ \\
Georgia Tech                          & Yearly cost: & \$27,872 & 20 year ROI: & \$\$\$
\end{tabular}%
}
\end{table}


\begin{multicols}{2}
% \begin{itemize}
%     \item{Undergraduate} 
%     \begin{itemize}
%         \item{Dream Schools}
%             \begin{itemize}
%                 \item{Massachusetts Institute of Technology}
%                     \begin{itemize}
%                         \item{Total 4 year cost (state):} \$232,000
%                         \item{20 year ROI:} \$962,000
%                     \end{itemize}
%                 \item{Stanford}
%                     \begin{itemize}
%                         \item{Total 4 year cost (state):} \$240,000 
%                         \item{20 year ROI:} \$854,000
%                     \end{itemize}
%             \end{itemize}
%         \item{Best Value Schools} 
%             \begin{itemize}
%                 \item{UT Austin}
%                     \begin{itemize}
%                         \item{Total 4 year cost (state):} \$86,000 
%                         \item{20 year ROI:} \$931,100
%                     \end{itemize}
%                 \item{Georgia Tech}
%                     \begin{itemize}
%                         \item{Total 4 year cost (state):} \$210,000
%                         \item{20 year ROI:} \$756,000
%                     \end{itemize}
%             \end{itemize}
%     \end{itemize}
%     \item{Graduate} 
%     \begin{itemize}
%         \item{Dream Schools}
%             \begin{itemize}
%                 \item{Massachusetts Institute of Technology}
%                     \begin{itemize}
%                         \item{Yearly cost (state):} \$46,000
%                     \end{itemize}
%                 \item{Stanford}
%                     \begin{itemize}
%                         \item{Yearly cost (state):} \$48,720
%                     \end{itemize}
%             \end{itemize}
%         \item{Best Value Schools} 
%             \begin{itemize}
%                 \item{UT Austin}
%                     \begin{itemize}
%                         \item{Yearly cost (state):} \$9,564 
%                     \end{itemize}
%                 \item{Georgia Tech}
%                     \begin{itemize}
%                         \item{Yearly cost (state):} \$27,872
%                     \end{itemize}
%             \end{itemize}
%     \end{itemize}
% \end{itemize}

\section{Degree Description}
    \subsection{Undergraduate}
        Many engineering students across different engineering fields take the same classes. All engineering fields are required to take Calculus and Physics. After those basic classes, it begins to differentiate. For example, in ECE, the student must take ECE, whereas a student in Aerospace must take Thermodynamics. Additionally, all students must take basic English and History classes, though many students test out of these classes through placement tests and AP exams (“Degree Plans and Requirements”). 
        Most people do internships, either paid and unpaid, while in college. Many big name companies offer internships, including NASA, SpaceX, and Microsoft. Some students also begin working in startup companies, or new companies trying to make their way into the market. 
    \subsection{Graduate}
        Not all engineers go to graduate school. Many choose to leave college after their bachelors degree and go straight to the industry. Meanwhile, others go to get either a masters or PhD. One thing to note is that most engineering students do not pay for graduate school. In fact, if the student is going for a PhD, it is extremely unlikely the student will pay anything in tuition (“Engineering Your Graduate School Experience”). This is because the school is funding the student for the research or teaching work that the student does. Eddie Machek, who is earning a master's degree in civil engineering from the University of Akron and who will start a doctoral program in engineering at Georgia Tech this fall, explains the difference between the degrees this way: \"At a bachelor's level you are going to go out and do what's been done. At the master's level you are going to be in charge of the people who are doing that stuff. In a Ph.D., that's a whole other thing because you are doing the new stuff. You are in a lab.\" Additionally, the value of getting masters or doctorate degrees depends on the sector of engineering one is going into. “Chemical engineering and biomedical engineering have more employment opportunities for Ph.D. students, she says. Civil engineering, on the other hand, has more employment openings outside academia for those with master’s degrees than for those with doctorates.”  (Haynie) 
        People in the Engineering world do research on all kinds of topics that affect society. Some examples of research studies include Smart Vehicle Tracking Algorithms, Emulating Correct by Construction Biological Organs, and Impacts of Large-Scale PV [PhotoVoltic Solar Panels] integration into Distribution Grid (“Current Research Projects”). 
\section{Interview}
        I emailed Drew Crenwelge to describe how he got to be the Senior Test Director for the Falcon 9 rocket project at SpaceX. Early in high school, Mr. Crenwelge wanted to become a marine biologist. As he progressed through high school, he entertained the idea of becoming an architect before deciding that he would go into engineering. After taking Principles of Engineering I \& II in his final years of high school, he was set on his mission to become an aerospace engineer, focusing on rocket propulsion. Mr. Crenwelge received his BS in Aerospace Engineering (Rocket Propulsion) in 2011, graduating with a mathematics minor as well. During his college years at Purdue University, he interned at NASA’s Johnson Space Center and later applied to internships at Boeing, Blue Origin, NASA, and SpaceX. He took the SpaceX internship and became a Test Engineer Intern where he got firsthand experience in the field. 5 years later, he now works a 60 hour workweek, with high stress (Crenwelge). However, he loves his job. 
        Drew Crenwelge gave me some tips for getting to where he is in life. Mr. Crenwelge says to study Mechanical or Aerospace Engineering at a college like TAMU, UT, Purdue, Stanford, or Baylor. He recommends taking computer science classes that teach Python and Matlab, as well as electrical engineering classes. Mr. Crenwelge also recommends interning during the summer at companies like SpaceX. He also says to take a leadership role in clubs and activities in high school and college. 
\section{Work-Life Balance}
    60 hour workweek seems fairly average; however, it is never boring. There is always work to be done and one will always feel behind. 
    Though it is very high stress, people in the engineering industry love their jobs. They always love talking about the cool things they are working on.
\end{multicols}