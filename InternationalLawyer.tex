\chapter{International Lawyer}

International law is a specialized, interdisciplinary field of law that focuses on any and all cross-border issues, and the field covers a wide range of subspecialties (Smith-Barrow). Sub-specialties can range from human rights to international business. The field is so diverse because of the various types of transactions that can occur between the U.S. and other countries (Smith-Barrow). Because international law is such a diverse field, there are several career paths for international lawyers. These paths involve working in law firms, federal government, corporate counsel, international organizations and non-profit groups (Smith-Barrow). International law not only requires the person to know the legal systems of the United States, but also other countries’ judicial courts and the political and cultural influences of their respective courts (Smith-Barrow). Other skills required in this field are strong analytical ability, research, and writing skills. Also, learning foreign languages is highly recommended (Smith-Barrow). The average salary of an international lawyer is \$114,970 (“International Law Majors: Salary and Career Facts”).

\section{Top Colleges}
\begin{itemize}
    \item{Undergraduate} 
    \begin{itemize}
        \item{Dream Schools}
            \begin{itemize}
                \item{Harvard}
                    \begin{itemize}
                        \item{Total 4 year cost (state):} \$280,400 
                        \item{20 year ROI:} \$756,350
                    \end{itemize}
                \item{Princeton}
                    \begin{itemize}
                        \item{Total 4 year cost (state):} \$235,860 
                        \item{20 year ROI:} \$800,000
                    \end{itemize}
            \end{itemize}
        \item{Best Value Schools} 
            \begin{itemize}
                \item{George Mason University}
                    \begin{itemize}
                        \item{Total 4 year cost (state):} \$188,976 
                        \item{20 year ROI:} \$847,764
                    \end{itemize}
                \item{Arizona State University}
                    \begin{itemize}
                        \item{Total 4 year cost (state):} \$178,536
                        \item{20 year ROI:} \$858,204
                    \end{itemize}
            \end{itemize}
    \end{itemize}
    \item{Graduate} 
    \begin{itemize}
        \item{Dream Schools}
            \begin{itemize}
                \item{Yale}
                    \begin{itemize}
                        \item{3 year cost (state):} \$240,423
                        \item{20 year ROI:} \$2,058,977
                    \end{itemize}
                \item{Harvard}
                    \begin{itemize}
                        \item{3 year cost (state):} \$265,800
                        \item{20 year ROI:} \$2,033,600
                    \end{itemize}
            \end{itemize}
        \item{Best Value Schools} 
            \begin{itemize}
                \item{University of Alabama}
                    \begin{itemize}
                        \item{3 year cost (state):} \$170,688 
                    \end{itemize}
                \item{UT Austin}
                    \begin{itemize}
                        \item{3 year cost (state):} \$163,800 
                        \item{20 year ROI:} \$2,135,600 
                    \end{itemize}
            \end{itemize}
    \end{itemize}
\end{itemize}

\section{Degree Description}
    \subsection{International Studies}
        International Studies is an interdisciplinary major in which students accumulate foreign language skills, political and cultural knowledge, a foundational economics base, written and oral communications skills, and research skills (“International Studies: Careers”). In most International Studies degree programs, schools require students to learn a foreign language, and participate in some form of study abroad program (“International Studies: Careers”). Because of their interdisciplinary base, International Studies majors develop a strong foundation for both business and law (“International Studies: Careers”). In the business world, IS majors are very useful because of the highly globalized market, and they have the ability to perform many tasks, such as global financial analysis, security analysts and financial advising. Similarly, IS majors are strong candidates for law school because of their diverse background. Students can also choose to work for international nonprofits, which work to provide services around the globe (“International Studies: Careers”). 
    \subsection{International Law}
        Because the scope of international law is so large, students can take a variety of classes to achieve that J.D. (Smith-Barrow.)For example, at UC-Berkeley, students can take the following classes: human rights and humanitarian law, international trade and international investment law (Smith-Barrow). Students choose which area they desire to specialize in. Several institutions recommend that students be fluent in at least one language other than English due to the international nature of this profession. Some languages that are very popular are Spanish, Russian, German, Mandarin and Arabic (Smith-Barrow). When choosing a law school, USNews recommends students seeks schools which help attendees locate internships and fellowships, have an international law journal (a student run publication which publish legal articles written by students and professional alike), and have a legal clinic where students can work on actual cases (Smith-Barrow). 

\section{Interview}
    Steven Freeland, who first was an investment banker for thirteen years and a then lawyer for seven years, decided to become an academic (Freeland). Due to his experience and knowledge of international transactions and multi-jurisdictional legal matters, International Law was the perfect fit (Freeland). In an interview with Survive Law, Freeland discusses the pros and cons of working for an international organization like the ICC (Freeland). For instance, he considers the establishment of the ICC to be a step in the right direction toward true accountability. However, a significant con comes at the result of the ICC being a bureaucratic organization, which can lead to frustrating situations (Freeland). When Freeman began discussing the study of the law, he recommended that students should take a wide range of courses because understanding the law is of the utmost importance. Then, once a strong foundation has been set, students can work more actively to become specialized in a particular field. He also recommends that students should learn a language, and read the international sections of the newspaper everyday (Freeland).
\section{Work-Life Balance}
    Typically, lawyers expect to work very long hours. Most law firms require at least 1,800 billable hours each year, which equals 36 billable hours per week (“How Much Do Associates Work?”). Depending on the size of the firm, associates can work between 50-100 hours per week. At smaller firms of 500 people or less, lawyers work as low as 50 hours a week, but in larger firms, they can work upwards of 100 hours (“How Much Do Associates Work?”). However, regardless of the size of the firm, lawyers have to put in a lot of work in the beginning of their careers, and will have to put in a lot of hours. Only when lawyers make partner at their firm can they begin to cut back on their hours. However, if you fail to make partner, you most likely have to cut ties with your current firm and start the process over again (“How Many Hours A Week Does A Lawyer Work”). Additionally, government jobs are also available to lawyers, which only require a 40-hour per week commitment. However, the positions that require less of a time commitment also pay less than the positions at a large firm (“How Many Hours A Week Does A Lawyer Work”). Perspective lawyers must assess their desires for the future, and must take into account their work ethic and desired lifestyle before they commit to being a lawyer.
